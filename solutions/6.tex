\section*{Blatt 6}
%

\subsection*{Aufgabe 1}

\begin{enumerate}[a)]
  \item Da $L$ eine reguläre Sprache ist, gibt es einen DEA $A = (Q, \Sigma, \delta, q_0, F)$, der $L$ akzeptiert. $A$ würde daher jeden Zustand $\bar{l} \in \bar{L}$ ablehnen. Daher können wir einen Automaten $\bar{A} = (Q', \Sigma', \delta', q'_0, Q-F)$ konstruieren, der genau die Zustände akzeptiert, die $A$ ablehnt. $\bar{A}$ akzeptiert also genau dann ein Wort $w$, wenn es $\bar{\delta}(q_0, w) \in Q-F$ gibt.

  \item Mit $L^2$ bezeichnen wir die Konkatenation von
   $L = L_1$ mit sich selbst und mit $L_i$ die Konkatenation
   von $L_i-1$ und $L$. Damit enthält $L_i$ genau die Wörter,
    die sich so in $i$ Teile zerlegen lassen, daß alle Teile
    zu $L$ gehören. Der positive Kleenesche Abschluß $L^{+}$
    besteht aus der Vereinigung aller $L_i$, $i \geq 1$. Wenn
     wir testen wollen, ob ein Wort zu $L^{+}$ gehört, müssen
     wir ein passendes $i$ und eine passende Zerlegung des Wortes
      ``raten''. Eine effiziente Konstruktion deterministischer
      Automaten scheint also ausgeschlossen zu sein, und wir gehen
       den Umweg über nichtdeterministische Automaten. Es genügt
       eine Kopie eines Automaten für $L$, in der wir in allen
       akzeptierenden Zuständen eine $\epsilon$-Bewegung zum Anfangszustand
        erlauben. Immer wenn eine derartige $\epsilon$-Bewegung
        ausgeführt wird, ist ein Teilwort aus $L$ gelesen. Es werden
        also genau die Wörter aus $L^{+}$ akzeptiert. Der eigentliche
         Kleenesche Abschluß $L^{*}$ von $L$ besteht aus der Vereinigung
         aller $L^{i}, i \geq 1$, wobei $L^{0}$ nur das leere Wort enthält.
         Da $L^{*}$ die Vereinigung von $L^{+}$
 und $\{\epsilon\}$ ist, erhalten wir auch leicht einen nichtdeterministischen
 Automaten für $L^{*}$.
\end{enumerate}

\subsection*{Aufgabe 2}

\begin{enumerate}[a)]
\item $L_1 = \{ w_1 \ldots w_n \in \Sigma^{*} \given{n\geq 1 \land (w_1 = 0 \lor w_n = 1)} \} $

Es gilt also entweder, dass 1 am Ende oder 0 am Anfang des Wortes steht. Mit $n=1$ kann das entstehende Wort $w_1 = w_n$ entweder 0 oder 1 sein. Wir können also schreiben
\[ L_1 = (0) \cup (0 \cup 1)^{*} \cup (1) \]

\item $L_2 = \{w_1, \ldots w_n \in \Sigma^{*} \given{n\geq 3 \; \text{und} \; \exists i \in \{ 1, \ldots, n-2 \} : w_i = w_{i+1} = w_{i+2} = 0 } \}$

Das heißt in jedem Wort ist 000 enthalten, wobei es auch hierauf enden darf. Wir können also schreiben:

\[ L_2 = (0 \cup 1)^{*} \, 000 \, (0 \cup 1)^{*} \]

\item $ L_3 = \{ w_1, \ldots w_n \given{ \forall i \in \{1, \ldots, n\}: \; w_i = 1 \implies (i<n \land w_{i+1} = 0) } \} $

Für jedes Wort gilt also $n \geq 2$, weil für $i$ auch $i+1$ als Index existieren muss. 10 ist daher zulässig, auf jede 1 muss eine 0 folgen, 1 darf nicht das Ende des Wortes bilden. Daher:

\[ L_3 = (0)^{*} \cup (10)^{*} \, 10  \]

\end{enumerate}




\subsection*{Aufgabe 4}
%
Gegeben $f: \N \times \N \to \N$ mit $f(x,y) = \dfrac{ (x+y) (x+y+1) }{ 2 } + y$. Wir wollen zeigen, dass $f$ bijektiv ist. Dazu zeigen wir, dass $f$ injektiv und surjektiv ist. \\
%
\begin{enumerate}[i)]
%
\item $f$ ist injektiv.\\
%
Dann muss für $ x_1, x_2, y_1, y_2 \in \N $ gelten, dass $f(x_1, y_1) = f(x_2, y_2)$ nur für $(x_1, y_1) = (x_2, y_2)$.
%
Sei $x_1 + y_1 > x_2 + y_2$, dann haben wir
%
\begin{align*}
f(x,y) &= \dfrac{(x_1 + y_1) (x_1 + y_1 +1)}{2} + y_1 \\
       &\geq \dfrac{(x_2 + y_2 + 1) (x_2 + y_2 + 2)}{2} \\
       &= \dfrac{(x_2 + y_2 + 1)(x_2 y_2)}{2} + x_2 + y_2 + 2 \\
       &> \dfrac{(x_2 + y_2 + 1)(x_2 y_2)}{2} + y_2 \\
       &= \dfrac{(x_2 + y_2)(x_2 + y_2 + 1)}{2} + y_2 = f(x_2, y_2) \\
\end{align*}
%
Da aus $(x_1, y_1) > (x_2, y_2)$ folgt, dass $f(x_1, y_1) > f(x_2, y_2)$ muss auch gelten, dass $f(x_2, y_2) > f(x_1, y_1) \iff (x_2, y_2) > (x_1, y_1)$.
%
Es kann daher nur $f(x_1, y_1) = f(x_2, y_2)$ gelten, wenn $(x_1, y_1) = (x_2, y_2)$. Für $(x_1, y_1) = (x_2, y_2)$ gilt.
%
\begin{align*}
  0 &= f(x_1, y_1) - f(x_2, y_2) \\
  &= \dfrac{(x_1 + y_1)(x_1 + y_1 + 1)}{2} + y_1 - \br{\dfrac{(x_1 + y_1)(x_1 + y_1 + 1)}{2} + y_1}\\
  &= y_1 - y_2
\end{align*}

%
für $y_1 = y_2$ und $x_1 = x_2$.
%
\item $f$ ist surjektiv. \\
%
Es muss also gelten, dass $\forall n \in \N, \exists x,y \in \N \times \N: \; n = f(x,y)$. \\
%
Induktionsanfang $n=0$: Für $x=y=0$ ist \[f(0,0) = \dfrac{(0+0)(0+0+1)}{2} + 0 = 0\] \\
%
Induktionsschluss $n \to n+1$: \\
%
Gilt $n = f(x,y)$, dann ist
%
%
\begin{align*}
n+1 &= f(x,y) + 1\\
    &= \dfrac{(x+y)(x+y+1)}{2} + y + 1\\
    &=
    \begin{dcases}
        \dfrac{(x-1+y+1)(x-1+y+1+1)}{2}+y+1 \\
        = f(-x, y+1) & x \in \N_{>0}\\[2em]
        \dfrac{y(y+1)}{2}+y+1 = \dfrac{(y+2)(y+1)}{2} \\
        = \dfrac{(y+1)(y+2)}{2} \\
        = f(y+1, 0) = f(y+1,x) & x=0 \\
    \end{dcases}
\end{align*}
\end{enumerate}

$f$ ist also injektiv und surjektiv und damit bijektiv.


      % automata stuff
      % \begin{tikzpicture}[>=stealth',shorten >=1pt,auto,node distance=2cm]
      % \node[initial,state,accepting]  (q0)     {$q_0$};
      % \node[state, accepting]  (q1) [right of=q0]  {$q_1$};
      % \node[state, accepting]  (q2) [right of=q1] {$q_2$};
      % \node[state, accepting]  (q3) [right of=q2]  {$q_3$};
      % \node[state, accepting]  (q4) [right of=q3] {$q_4$};

      % \path[->] (q0) edge [loop above] node {0} (q0)
      %            edge              node {1} (q1)
      %       (q1) edge [loop above] node {0} (q1)
      %            edge              node {1} (q2)
      %       (q2) edge [loop above] node {0} (q2)
      %            edge              node {1} (q3)
      %       (q3) edge [loop above] node {0} (q3)
      %            edge              node {1} (q4)
      %       (q4) edge [loop above] node {0} (q4)
      %       ;
      % \end{tikzpicture}



      % more automata stuff
      % \begin{tikzpicture}[>=stealth',shorten >=1pt,auto,node distance=2cm]
      % \node[initial,state,accepting]  (q0)     {$q_0$};
      % \node[state, accepting]  (q1) [right of=q0]  {$q_1$};


      % \path[->] (q0) edge [loop above] node {1} (q0)
      %            edge              node {0} (q1)
      %       (q1) edge [bend left]  node {1} (q0)

      %       ;
      % \end{tikzpicture}
