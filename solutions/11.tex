\section*{Blatt 11}
%

\subsection*{Aufgabe 1}

Wähle $p=5$ und $q=7$. Daher $n = 5 \cdot 7 = 35$. Wir haben
\[
  \phi(n) = \phi(35) = \phi(5 \cdot 7) = \phi(5) \cdot \phi(7) = 6 \cdot 4 = 24
\]

Verschlüssele $n \in \{0, \ldots, 34\}$. Wähle $e = 5$ ($1 < e < \phi(35)$  und $\text{ggT}\, (5,24) = 1$).

Suche $d \in \N$ so, dass $ed = 1 \mod \phi(n)$.
\begin{center}
\begin{tabular}{c c c
}

$d$ & $de$ & $de \; \text{mod}\; 24$ \\
\hline
1 & 5 & 5 \\
2 & 10 & 10 \\
3 & 15 & 15 \\
4 & 20 & 20 \\
5 & 25 & 1 \\
6 & 30 & 6 \\
7 & 35 & 11 \\
& \vdots & \\
28 & 140 & 20 \\
29 $ (=: d) $ & 145 & 1 \\
30 & 150 & 6 \\

\end{tabular}
\end{center}

Öffentlicher Schlüssel $(n, e) = (35,5)$ \\
Privater Schlüssel $(n,d) = (35, 29)$ \\

Nachricht \texttt{FELIX} = \{ 6 5 12 9 24 \} (Nummer im Alphabet).

\begin{center}
  \begin{tabular}{ccc}
    & x & $x^5 \mod 35$ \\
    \hline
    F & 6 & 6 \\
    E & 5 & 10 \\
    L & 12 & 17 \\
    I & 9 & 4 \\
    X & 24 & 19 \\
  \end{tabular}
\end{center}

Geheimtext: \{ 6 10 17 4 19 \}

\subsection*{Aufgabe 2}

\[
  \phi = ((x_1 \implies \neg x_2) \land \neg(x_3 \iff x_1))
\]
\textbf{b)} \\
\begin{align*}
  \phi = ((x_1 \implies x_2) &\land \neg ((x_3 \implies x_1) \land (x_1 \implies x_3))) && \text{Elimination der Äquivalenz} \\
  \phi = ((\neg x_1 \lor \neg x_2) \land &\underbrace{ \neg ((\neg x_3 \lor x_1) \land (\neg x_1 \lor x_3)))} && \text{Elimination der Implikation}\\
%
  &= \neg (\neg x_3 \lor x_1) \lor \neg (\neg x_1 \lor x_3) && \text{De Morgan} \\
%
  &= (\neg \neg x_3 \land \neg x_1) \lor (\neg \neg x_1 \land \neg x_3) && \text{De Morgan} \\
%
  &=(x_3 \land x_1) \lor (x_1 \land \neg x_3) && \text{Elimination der doppelten Negation} \\
%
  \phi = ((\neg x_1 \land \geq x_2) &\land \underbrace{((x_3 \land \neg x_1) \lor (x_1 \land \neg x_3)))} && \text{(*)}\\
%
  &= ((x_3 \land x_1) \lor x_1) \land ((x_3 \land \neg x_1) \lor \neg x_3) \\
%
  &= ((x_3 \lor x_1) \land (\neg x_1 \lor x_1)) \land ((x_2 \lor \neq x_3) \land (\neg x_1 \lor \neg x_3))\\
%
\end{align*}
%
\begin{align*}
  \phi = ((\neg x_1 \lor \neg x2) &\land ((x_3 \lor x_1) \land (\neg x_1 \lor x_1) \land ((x_3 \lor \neg x_3) \land (\neg x_1 \lor \neg x_3)))) \\
\end{align*}

Mit Abkürzungen aus Roeglin 12/13, S.84 folgt

\begin{align*}
  \phi = (\neg x_1 \lor \neg x_2) &\land (x_3 \lor x_1) \land \underbrace{(\neg x_1 \lor x_1)}_{\text{immer wahr}} \land \underbrace{(x_3 \lor \neg x_3)}_{\text{immer wahr}} \land (\neg x_1 \lor \neg x_2) \\
  %
  \phi = (\neg x_1 \lor \neg x_2)& \land (x_3 \lor x_1) \land (\neg x_1 \lor \neg x_3) \\
\end{align*}

\textbf{a)} Benutze $(*)$ für Wahrheitstabelle:

\begin{tabular}{ccc|ccc|cc}
$x_1$ & $x_2$ & $x_3$ & $A = (\neg x_1 \lor \neg x_2)$ & $B = (x_3 \land \neg x_1)$ & $C = (x_1 \land \neg x_3)$ & $D=(B \lor C)$ & $\phi = (A \land D)$ \\
\hline

0 & 0 & 0 & 1 & 0 & 0 & 0 & 0 \\
1 & 0 & 0 & 1 & 0 & 1 & 1 & 1 \\
0 & 1 & 0 & 1 & 0 & 0 & 0 & 0 \\
1 & 1 & 0 & 0 & 0 & 1 & 1 & 0 \\
0 & 0 & 1 & 1 & 1 & 0 & 1 & 1 \\
1 & 0 & 1 & 1 & 0 & 0 & 0 & 0 \\
0 & 1 & 1 & 1 & 1 & 0 & 1 & 1 \\
1 & 1 & 1 & 0 & 0 & 0 & 0 & 0 \\
\hline
\end{tabular}

\vspace{1em}
DNF: \[
  (x_1 \land \neg x_2 \land \neg x_3) \lor (\neg x_1 \land \neg x_2 \land x_3) \lor (\neg x_1 \land x_2 \land x_3)
\]