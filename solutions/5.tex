\section*{Blatt 5}

\subsection*{Aufgabe 1}

\begin{enumerate}[a)]

\item Für $n\geq 2$ gilt $\prod_{k=2}^n (1 - \frac{1}{k}) = \frac{1}{n}$ \\

I.A.: $n=2$. $1 - \frac{1}{2} = 1 - \frac{1}{2} \; \checkmark$

Induktionsschritt $n \to n+1$:

\begin{align*}
\prod_{k=2}^{n+1} (1 - \frac{1}{k}) &= \frac{1}{n+1} \\
(1 - \frac{1}{n+1}) \prod_{k=2}^{n} (1 - \frac{1}{k}) &= \frac{1}{n+1} \\
\text{IV anwenden:} \; (1- \frac{1}{n+1})\frac{1}{n} &= \frac{1}{n+1} \\
\frac{(n+1)-1}{n+1} \frac{1}{n} &= \frac{1}{n+1}\\
\frac{n}{n+1} \frac{1}{n} &= \frac{1}{n+1}\\
\frac{n}{(n+1)n} &= \frac{1}{n+1}\\
\frac{1}{n+1} &= \frac{1}{n+1} \hspace{2em} \blacksquare
\end{align*}

\item Für $n\in\N$ gilt $\prod_{k=1}^{n} (1 + \frac{2}{k}) = \sum_{i=1}^{n+1} i$\\

I.A.: $n=1$, $ 1 + \frac{2}{1} = 1+2 \; \checkmark$ \\

Induktionsschritt $n \to n+1$:

\begin{align*}
\prod_{k=1}^{n+1} \bp{1 + \frac{2}{k}} &= \bp{\sum_{i=1}^{n+1} i} + (n+2) \\
1 + \frac{2}{n+1} \prod_{k=1}^{n} \bp{1 + \frac{2}{k}} &= \bp{\sum_{i=1}^{n+1} i} + n + 2 \\
\text{I.V.:} \; \bp{ 1 + \frac{2}{n+1} } \bp{ \sum_{i=1}^{n+1} i } &= \sum_{i=1}^{n+1} + n + 2\\
1 + \frac{2}{n+1} &= \frac{n+2}{ \sum_{i=1}^{n+1} +1}\\
\text{Gauss:} \; \frac{2}{n+1} + 1 &= \frac{2(n+2)}{(n+1)(n+2)}+1\\
\frac{2}{n+1} &= \frac{2}{n+1} \hspace{2em} \blacksquare
\end{align*}



\end{enumerate}

\subsection*{Aufgabe 3}
  Zeigen Sie mithilfe des Pumping-Lemmas, dass die folgenden Sprachen über dem
Alphabet $\Sigma = \{a, b, c\}$ nicht regulär sind. \\

Pumping Lemma:
\[ \exists n \in \N_0, \forall z \in L, \abs{z} \leq n: \;
 \exists u,v,w \; \text{mit} \; z = uvw \]

Für die Zerlegung von $z$ gilt dann:
\begin{enumerate}
\item $\abs{uv} \leq n$
\item $ \abs{v} > 0$
\item $\forall \, i \in \N_0: \; uv^i w \in L$
\end{enumerate}

\vspace{1em}

  \begin{enumerate}[a)]
  \item $ L_1 = \{ a^i b^{2i} \given{i \in \N} \} $ \\

    Angenommen $L_1$ sei regulär. Dann gibt es nach dem Pumping Lemma eine Zahl $n$, so dass sich alle Wörter $ z \in L_1 $ mit $ \abs{z} \geq n $ wie folgt zerlegen lässt.

    Betrachte speziell das Wort $z = uvw = a^n b^{2n}$.

    Gemäß Bedingung 2 ist $v$ nicht leer, gemäß Bedingung 1 besteht $uv$ und somit auch $v$ ausschließlich aus $a$s (da $\left|uv\right| \leq n$ und $\left|uvw\right| = \left|a^n b^{2n}\right| = 3n)$. Mit Bedingung 3 müsste das Wort
    \[ uv^2w = a^{n - \left|v\right|} a^{2 \cdot \left|v\right|}b^2n = a^{n+ \left|v\right|}b^2n \]
    in $L$ liegen. Das ist aber offensichtlich falsch, denn dieses Wort hat mehr als halb soviele $a$s als $b$s, da  ${\left|v\right|}$ größer 0 und damit $\abs{n} + \abs{v} > \abs{n}$ . Damit gilt: $L$ kann nicht regulär sein. \\

  \item $ L_2 = \{  a^{i} b^{j} c^{k} \given {i,j,k \in \N \; \text{und} \; i < j < k}  \} $ \\

    Wähle $z = uvw = a^{\alpha} b^{\beta} c^{\gamma}$, sodass $\alpha < \beta < \gamma$ gilt, nach Voraussetzung. Es gilt dann $\abs{uvw} = \abs{ a^{\alpha} b^{\beta} c^{\gamma} } = \alpha + \beta + \gamma $ und $\abs{uv} \leq n$, da $\alpha + \beta \leq \alpha + \beta + \gamma =: n$. Außerdem gilt $v > 0$.

    Daher muss mit (3) auch das Wort $z^{\star} := uv^2 w = a^{\alpha} (b^{\beta})^2 c^{\gamma} = a^{\alpha} b^{2 \beta} c^{\gamma}$ in $L_2$ liegen. Dies ist jedoch nicht der Fall, da für $z^{\star}$ nicht mehr gilt, dass $\alpha < 2 \beta < \gamma$. Dies wird offensichtlich, wenn man bspw. $\alpha = 1, \; \beta = 2, \; \gamma = 3$ wählt. Daher gilt $z^{\star} \notin L_2$ und somit ist $L_2$ nicht regulär.\\

  \item $ L_3 = \{  w \in \Sigma^{*} \given{ \abs{w} \; \text{ist eine Zweierpotenz} }  \} $ \\

  Definiere $L_4 := \{  a^{2^{k}} \given{k \in \N}  \}$. Es gilt $L_4 \subset L_3$.

  Wähle nun $z = a^{2^{k}}$. Es gilt offensichtlich $z \in L_4$ und daher auch $z \in L_3$ sowie $\abs{z} \geq n$.

  Zerlege $z$ in $uvw$ so, dass $\abs{uv} \leq n$ und $\abs{v} > 0$ mit

  \[  u = a^p, \; v = a^q, \; w = a^{2^{n} - p -q} \given{p+q \leq n, \; q > 0} . \]

  Sei oBdA $i=2$, dann gilt $uv^{i}w = a^{2^{n} + q}$. Aufgrund von $2^n \geq n, \; \forall \, n \in \N$ folgt, dass $p+q < 2^n$ und daher $0 < q < 2^n$. Das heisst

  \[  2^n < 2^n + q < 2^n + 2^n = 2 \cdot 2^n = 2^{n+1}  \]

  Daher ist $2^n + q$ keine Zweierpotenz, sondern liegt zwischen $2^n$ und $2^{n+1}$. Somit folgt $z = uv^{i}w = uv^{2}w \notin L_4$ und $z \notin L_3$. Also verletzen sowohl $L_4$ als auch $L_3$ das Pumping Lemma und sind nicht regulär.
  \end{enumerate}


\subsection*{Aufgabe 4}

Gegeben seien zwei Sprachen $L_1 = \{
0^{k}
 1^{l}
\given{k,l
\geq 0}\}$
und
$L_2 = \{1^k 0^l 1^l
\given{k,l \geq 1} \}$ über dem
 Alphabet $\{0, 1\}$
  sowie deren Vereinigung
  $L =
  L_1
  \cup
  L_2$.
\begin{enumerate}[a)]

\item Zeigen Sie, dass man mithilfe des Pumping-Lemmas nicht zeigen kann, dass es keinen DFA gibt, der die Sprache $L$ entscheidet.\\

Prüft man $L$ auf die Bedingungen des Pumping Lemmas sollte $L$ dieses nicht erfüllen, insofern $L$ nicht regulär ist. Allerdings erfüllt $L$ das Pumping Lemma, weil für jedes Wort $z \in L$ eine Zerlegung $uvw$ existiert mit $\abs{uv} \leq n, \; v>0$ und $\abs{z} geq n$ für die auch $uv^iw \in L$ für $l \geq 0$.

Dazu kann $v$ einfach als erster Buchstabe gewählt werden. Dieser ist entweder ein $a$, die Anzahl von führenden $a$s ist beliebig. Oder er ist ein $b$ oder $c$, ohne führende $a$s ist aber die Anzahl von führenden $b$s oder $c$s beliebig.




\item Zeigen Sie, dass die Sprache L nicht von einem DFA entschieden werden kann.\\

Die Forderung, dass $L$ nicht von einem DFA entschieden kann ist analog zur Forderung, dass $L$ regulär ist.

Betrachte erst die Sprache $L_0 = \{  0^k 1^k \given{k \geq 0} \}$. Wenn ein DFA für die Wörter $w$ und $w'$ denselben Zustand erreicht, erreicht er auch für jedes Wort $w''$ auch für die Wörter $ww''$ und $w'w''$ denselben Zustand. Das heisst er akzeptiert $ww''$ und $w'w''$ oder er akzeptiert beide nicht. Nimm an, dass der DFA $A$ die Sprache $L_0$ entscheidet.

Betrachte nun die unendlich vielen Wörter $w_k = 0^k, \; k \geq 1$. Da ein DFA nur endlich viele Zustände haben darf, gibt es zwei Wörter $w_i$ und $w_j$ mit $i \neq j$ so, dass $A$ nach dem Lesen von $w_i$ und $w_j$ den gleichen Zustand erreicht. Für $w'' = 1^i$ ist $w_i w'' \in L_0$ und $w_j w'' \notin L_0$. $A$ trifft also für eines der Wörter $w_i w''$ oder $w_j w''$ eine falsche Entscheidung. Daher gibt es keinen DFA, der $L_0$ entscheidet.

Betrachten wir nun $L = L_1 \cup L_2$ für verschiedene $k, l$:

\[ L=
\begin{cases}
(0^k 1^l) & l,k = 0 \\
            0 & l = 0, k=1 \\
            1 & l = 1, k=0 \\
(1^l 0^l 0^l) & l \geq 1, k=0 \\
    \underbrace{(0^k 1^k)}_{\equiv L_0} & l=0, k \geq 1 \\
(0^k 1^l 1^k 0^l 1^l) \iff (\underbrace{0^k 1^k}_{\equiv L_0} \underbrace{1^l 0^l}_{\equiv L_0} 1^l) & l,k \geq 1 \\
\end{cases}
\]

Wir sehen also, dass $L$ unter nicht trivialen Variablenbelegungen äquivalent zu solchen Sprachen ist, die nicht durch DFAs darstellbar sind. Daher ist $L$ selbst nicht regulär.


\end{enumerate}













      % automata stuff
      % \begin{tikzpicture}[>=stealth',shorten >=1pt,auto,node distance=2cm]
      % \node[initial,state,accepting]  (q0)     {$q_0$};
      % \node[state, accepting]  (q1) [right of=q0]  {$q_1$};
      % \node[state, accepting]  (q2) [right of=q1] {$q_2$};
      % \node[state, accepting]  (q3) [right of=q2]  {$q_3$};
      % \node[state, accepting]  (q4) [right of=q3] {$q_4$};

      % \path[->] (q0) edge [loop above] node {0} (q0)
      %            edge              node {1} (q1)
      %       (q1) edge [loop above] node {0} (q1)
      %            edge              node {1} (q2)
      %       (q2) edge [loop above] node {0} (q2)
      %            edge              node {1} (q3)
      %       (q3) edge [loop above] node {0} (q3)
      %            edge              node {1} (q4)
      %       (q4) edge [loop above] node {0} (q4)
      %       ;
      % \end{tikzpicture}




      % more automata stuff
      % \begin{tikzpicture}[>=stealth',shorten >=1pt,auto,node distance=2cm]
      % \node[initial,state,accepting]  (q0)     {$q_0$};
      % \node[state, accepting]  (q1) [right of=q0]  {$q_1$};


      % \path[->] (q0) edge [loop above] node {1} (q0)
      %            edge              node {0} (q1)
      %       (q1) edge [bend left]  node {1} (q0)

      %       ;
      % \end{tikzpicture}
