\section*{Blatt 7}
%

\subsection*{Aufgabe 1}
\begin{enumerate}[a)]
\item
\begin{align*}
 (1001)^{20} &= (1001)^{16} \cdot (1001)^{4}\\
             &= (1001)^{2^{4}} \cdot (1001)^{2^{2}}\\
\end{align*}

Rechnen wir die einzelnen Terme aus:

\begin{align*}
  (1001)^2 &= 2001 \mod 100000 \\
  (1001)^{2^{2}} = ((1001)^2)^2 = 2001^2 &= 4001 \mod 100000 \\
  (((1001)^2)^2)^2 = 4001^2 &= 8001 \mod 100000 \\
  (1001)^{2^{4}} = ((((1001)^2)^2)^2)^2 = 8001^2 &= 16001 \mod 100000 \\
\end{align*}

Daher können wir schreiben
\[  (1001)^{20} = (1001)^{2^{4}} \cdot (1001)^{2^{2}} \underbrace{=}_{\text{mit mod}} 8001 \cdot 16001 = 24001 \mod 100000 \]

Dies sind die letzten 5 Stellen der Dezimaldarstellung.


\item

Aufgrund von \[ (1+x)^n = 1 + \binom{n}{1}x + \binom{n}{2}x^2 + \cdot + \binom{n}{n} x^n  \]

haben wir $p_4 = \binom{n}{4} $ und $q_2= \binom{n}{2}  $.

Daher lösen wir

\begin{align*}
p_4 = 6q_2 &\iff \binom{n}{4} = 6 \binom{n}{2} \\
           &\iff \frac{n!}{ (n-4)! 4! } =  6 \frac{n!}{(n-2)! 2!}\\
           &\iff \frac{n!}{24(n-4)!} = \frac{3n!}{(n-2)!} \\
           &\iff \frac{1}{24} (n-3)(n-2)(n-1)n = 3(n-1)n \\
           &\iff n^2 + 11 = 12n \\
           &\implies n = \frac{12 \pm \sqrt{144 - 44}}{2} \\
           &n = 11
\end{align*}

Alle anderen möglichen Lösungen sind wegen Vorraussetzung $n\geq 2$ ungültig.

\end{enumerate}



      % automata stuff
      % \begin{tikzpicture}[>=stealth',shorten >=1pt,auto,node distance=2cm]
      % \node[initial,state,accepting]  (q0)     {$q_0$};
      % \node[state, accepting]  (q1) [right of=q0]  {$q_1$};
      % \node[state, accepting]  (q2) [right of=q1] {$q_2$};
      % \node[state, accepting]  (q3) [right of=q2]  {$q_3$};
      % \node[state, accepting]  (q4) [right of=q3] {$q_4$};

      % \path[->] (q0) edge [loop above] node {0} (q0)
      %            edge              node {1} (q1)
      %       (q1) edge [loop above] node {0} (q1)
      %            edge              node {1} (q2)
      %       (q2) edge [loop above] node {0} (q2)
      %            edge              node {1} (q3)
      %       (q3) edge [loop above] node {0} (q3)
      %            edge              node {1} (q4)
      %       (q4) edge [loop above] node {0} (q4)
      %       ;
      % \end{tikzpicture}



      % more automata stuff
      % \begin{tikzpicture}[>=stealth',shorten >=1pt,auto,node distance=2cm]
      % \node[initial,state,accepting]  (q0)     {$q_0$};
      % \node[state, accepting]  (q1) [right of=q0]  {$q_1$};


      % \path[->] (q0) edge [loop above] node {1} (q0)
      %            edge              node {0} (q1)
      %       (q1) edge [bend left]  node {1} (q0)

      %       ;
      % \end{tikzpicture}
