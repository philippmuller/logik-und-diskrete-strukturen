\section*{Blatt 9}
%

\subsection*{Aufgabe 1}

\begin{itemize}
  \item $ (\Z / 5 \Z, \odot) $
  \begin{enumerate}[a)]
  \item
    \begin{tabular}{c|ccccc}
    $\odot$ & 0 & 1 & 2 & 3 & 4 \\
    \hline
    0 & 0 & 0 & 0 & 0 & 0 \\
    1 & 0 & 1 & 2 & 3 & 4 \\
    2 & 0 & 2 & 4 & 1 & 3 \\
    3 & 0 & 3 & 1 & 4 & 2 \\
    4 & 0 & 4 & 3 & 2 & 1 \\
    \end{tabular}
    \item Die Auswertungsreihenfolge ist egal, dies ist an der Verknüpfungstafel leicht ersichtlich, daher haben wir Assoziativität. Kommutativität folgt aus der Symmetrie der Verknüpfungstafel gegenüber ihrer Diagonale. (f)

    Aus Übung:
    Assoziativität:
    \begin{align*}
      (\dbr{a} \odot \dbr{b}) \dbr{c} &= \dbr{a} \odot (\dbr{b} \odot \dbr{c}) \\
      \iff (\dbr{a \cdot b}) \odot \dbr{c} &= \dbr{a} \odot \dbr{b \cdot c} \\
      \iff \dbr{a \cdot b \cdot c}_n &= \dbr{a \cdot b \cdot c}_n \\

    \end{align*}
    \item Invertierbare Elemente sind \dbr{1}, \dbr{2}, \dbr{3}, \dbr{4}

    \end{enumerate}



  \item $ (\Z / 6 \Z, \odot) $
  \begin{enumerate}[a)]
  \item
  \begin{tabular}{c|cccccc}
    $\odot$ & 0 & 1 & 2 & 3 & 4 & 5\\
    \hline
    0 & 0 & 0 & 0 & 0 & 0 & 0 \\
    1 & 0 & 1 & 2 & 3 & 4 & 5 \\
    2 & 0 & 2 & 4 & 0 & 2 & 4 \\
    3 & 0 & 3 & 0 & 3 & 0 & 3 \\
    4 & 0 & 4 & 2 & 0 & 4 & 2 \\
    5 & 0 & 5 & 4 & 3 & 2 & 1 \\

    \end{tabular}

    \item analog zu $\Z / 5\Z$

    \item \dbr{1}, \dbr{5}

    \end{enumerate}


\end{itemize}

\subsection*{Aufgabe 2}
$(R, +, \cdot)$ kommutativer Ring mit $0$ als neutrales Element der Addition und $1$ als neutrales Element der Multiplikation und $1 \in R$ und $1 \neq 0$.

\begin{enumerate}[a)]
  \item Zu zeigen: $a \in R: (-1) \cdot a = -a$.

  Sei $r$ das neutrale Element zur Multiplikation. Dann können wir mit $a \in R$ umformen:
  \[ (-1) \cdot a = (-n) \cdot a \]
  Weil $(R, +, \cdot)$ kommutativer Ring ist können wir wir schreiben
  \[
    = -(n) \cdot a = -n \cdot a = -a
  \]
  Die letzte Gleichheit folgt aus der Definition des neutralen Elements. \\

  \item Zu zeigen: $  a,r \in R^{*}: \; a \cdot r = a \implies r = 1$.

  Wir wissen aus $a, r \in R^{*}$, dass $a$ und $r$ invertierbar sind. Daher haben wir die Eindeutigkeit, die in c) fehlen wird, da $0 \notin R^{*}$. $r$ muss dann also das neutrale Element darstellen, wenn $a \cdot r = a$ gelten soll. Aus der Aufgabenstellung wissen wir, dass das neutrale Element der Multiplikation eben 1 ist, daher gilt $r=1$. \\

  \item Zu zeigen: $  a,r \in R^{*}: \; a \cdot r = a \notimplies r = 1$.

  Betrachten wir $a=0$ so gilt für jedes $r$, dass $a \cdot r = a$. Somit können wir nicht $r=1$ für alle $a, r \in R$ schließen.\\

  \item Zu zeigen: $a, b \in R: \; a \cdot b = 0 \notimplies a + 1 \lor b+1 \; \text{sind Einheit}$.

  Würden wir $R = \Z$ wählen würde dies halten, da auf $\Z$ keine Nullteiler existieren (also solche $a \neq 0$ für die es ein $b$ gibt so, dass $a \cdot b = 0$).

  Da diese über beliebigen Körpern existieren können wir von $a \cdot b = 0$ nicht folgern, dass $a+1$ oder $b+1$ Element sind, wie dies möglich wäre, wenn wir wüssten, dass eines der beiden gleich 0 sein muss und 1 Element ist.\\

  \item $a \in R: \; a \cdot a = 0 \implies a+1$ ist Einheit.

  Die einzige Möglichkei für $a \cdot a = 0$ ist $a = 0$, da Nullteiler nur für $a, b: \; a \neq b$ existieren. Es gilt $0+1 = 1$ und $1 \cdot 1 = 1$, es ist also $x \cdot r = e$ erfüllt und es gilt daher, dass $1 \in R^{*}$.\\





\end{enumerate}
\subsection*{A2 Übung}

\begin{enumerate}
  \item 2a
    \begin{align*}
      -a = (-1)a \\
      = a\cdot 0 = 0\\
      \text{abelsche Gruppe} \\
      a + (-a) \ldots a = 0 \\
      (-1)a \; \text{additive Inverse von $a$}\\
      \implies (-1 a = -a) \\

    \end{align*}
    und

    \item 2b
    \begin{align*}
      ar = a \implies r = a \; a \in R^{*}\\
      r = 1 r = a^{-1}a r = a^{-1}a = 1\\
    \end{align*}

    \item 2d

    \begin{align*}
      ab = 0 \notimplies a+1 \lor b+1 \; \text{Einheit} \\
      R = \Z / 6 \Z, \; a = \dbr{2}, b = \dbr{3} \; \text{Gegenbeispiel}
    \end{align*}

    \item 2e

    aa = 0 $\implies$ $a \oplus 1$ ist Einheit.

    $a + 1 \in R^{*}$ wenn $b \in R$ ex mit $b \odot (a+1) = 1 \lor (a+1)\odot b = 1 $. Wir wählen $b = ((-a) + 1) \in R$, da $-a \in R$ und $R$ additiv abgeschlossen.

    \begin{align*}
      (a+1)\cdot ((-a) + 1) \\
      = a (-1) + a \cdot 1 + 1 \cdot ( -a) + 1 1 1 \\
      = (-1) (aa) + a + (-a) \\
      = 0 + a + (-a) + 1 = 1 \implies a + 1 \in R^{*}\\
    \end{align*}


\end{enumerate}


\subsection*{Aufgabe 3}
$x_0 := 8778, x_1 := 3230 \\$
\begin{enumerate}[a)]
  \item Primfaktorzerlegung:
  Teile durch Primzahlen beginnend mit 2. Rekursiv angewendet ergibt sich:
\begin{align*}
8778 &= 2 \cdot 3 \cdot 7 \cdot 11 \cdot 19 \\
3230 &= 2 \cdot 5 \cdot 17 \cdot 19 \\
\end{align*}
Gemeinsame Primfaktoren: $2$ und $19$, also ist $ggT(8778, 3230) = 2 \cdot 19 = 38$.
  \item Euklid:
\begin{align*}
x_{i+1} :&= x_{i-1} \; \text{mod} \; x_i \\
x_2 &= 8778 \; \text{mod} \; 3230 = 2318 \\
x_3 &= 912 \\
x_4 &= 494 \\
x_5 &= 418 \\
x_6 &= 76 \\
x_7 &= 38 \\
x_8 &= 0 \\
\end{align*}
Alos ist $ggT(8778, 3230) = x_7 = 38$.
\end{enumerate}

\subsection*{Aufgabe 4}

Wenn wir den euklidischen Algorithmus auf $f_{k+1}, f_k$ anwenden, also $\text{ggT}(f_{k+1}, f_k)$ suchen haben wir

\begin{align*}
f_{k+1} &= 1 \cdot f_k + f_{k-1} \\
f_{k}   &= 1 \cdot f_{k-1} + f_{k-2} \\
f_{k-1} &= 1 \cdot f_{k-2} + f_{k-3} \\
        &\vdots \\
f_4 &= 1 \cdot f_3 + f_2 \\
f_3 &= 2 \cdot f_2 + 0 \\
\end{align*}

Der letzte Term gilt, da $3 = f_3 = 2 \cdot f_2$, da $f_2 = 1$. Damit terminiert der Algorithmus nach $f_3$ und wir haben $k+1-2 = k-1$ Iterationen.\\

\textbf{Aus Übung:}\\

Beweis per Induktion nach $k$. I.A. $k=2$ also Eingabe $(x_0, x_1) = (f_3, f_2)$.

Im ersten durchlauf der while-Schleife wird $x_2 = x_0 \mod x_1$ berechnet. Dies ist gleich $f_3  \mod f_2$.

Es gilt $f_3 = 2$ und $f_2 = 1$. Daher $x_3 = 2 \mod 1=0$. Damit ist das Abbruchkriterium erfüllt. D.h. $(k-1) = 2-1 = 1$ Durchlauf wurde benötigt.

Sei $k>2$ beliebig. Wir betrachten den Aufruf des Alg. mit $(x_0, x_1) = (f_{k+2}, f_{k+1})$. Im ersten Durchlauf wird $x_2 = x_0 \mod x_1= f_{k+1} \mod f_{k+1}$ berechnet. Aus Def. der Fibonacci Zahlen folgt, dass $f_{k+2} = 1 \cdot f_{k+1} + f_k$. Damit folgt für $k>2$:
\[
   f_{k}>f_{k-1}: \; x_2 = f_{k+2} \mod f_{k+1} = f_k \\
 \]
 Im nächsten Durchlauf der Schleife wird $x_3 = f_{k+1} \mod f_k $ berechnet. Algorithmus wird auf $(x_0, x_1) = (f_{k+1}, f_k)$ angewandt.

 Aus Induktionsvoraussetzung wissen wir, dass die while-Schleife $(k-1)$ mal durchlaufen wird.

 Zusammen mit dem ersten Durchlauf ergeben sich dann $k$ Durchläufe, womit die Aussage bewiesen ist.