\section*{Blatt 4}

	\subsection*{Aufgabe 1}
  $R$ ist eine Äquivalenzrelation $\iff$ $R$ ist \emph{reflexiv}, \emph{symmetrisch} und \emph{transitiv}.
  \begin{enumerate}[a)]
  \item $R: \forall a_1,a_2 \in A: (a_1 \sim a_2 \iff f(a_1)=f(a_2))$ ist Äquivalenzrelation. \\

    \begin{enumerate}[i)]
      \item reflexiv: $\forall a \in M: aRa$. Es gilt $a_1 \sim a_2 \iff f(a_1) = f(a_2) \checkmark$
      %
      \item symmetrisch: $ \forall a, b \in M: (aRb \implies bRa)$. \\
      %
      \[ aRb \implies a \sim b \iff f(a) = f(b) \implies f(b) = f(a) \iff b \sim a \iff bRa  \checkmark \]
      %
      \item transitiv: $ \forall a,b,c \in M: ((aRb \land bRc) \implies aRc)  $:\\
      \[  (aRb \implies a \sim b \iff f(a) = f(b)) \land (bRc \implies b \sim c \iff f(b) = f(c))  \]
      Wegen $f(a) = f(b)$ und $f(b) = f(c)$ gilt $f(a)=f(c)$ und somit $a \sim c \iff aRc$. $\checkmark$ \\
      %
      Somit ist $R$ Äquivalenzrelation.
      %
    \end{enumerate}
    %
  \item Ist $\sim$ eine beliebige Äquivalenzrelation auf $A$ und ist $C = \{a_{\sim} \given{ a \in A} \}$ die Menge der Äquivalenzklassen von $\sim$, so gibt es eine Abbildung $p: A \to C$, so dass für alle $a_1, a_2 \in A$:\\
    \[a_1 \sim a_2 \iff p(a_1) = p(a_2)\]
    %
    %
  Da $\sim$ eine Äquivalenzrelation auf $A$ ist, ist sie reflexiv, symmetrisch und transitiv. Weil in $C$ alle Äquivalenzklassen von $\sim$ enthalten sind, welche aufgrund der Definition der Äquivalenzrelation o.g. Eigenschaften besitzen musst es für zwei Elemente $a, b$, welche $a \sim b$ erfüllen auch solche $p(a), p(b) \in C$ geben, so dass $p(a) = p(b)$ gilt. \\
  %
  Die Äquivalenzklassen zu zwei Elementen sind entweder gleich oder disjunkt, ersteres genau dann, wenn die Elemente äquivalent sind. Es gilt:
  %
  \[ [a_1] = [a_2] \iff a_1 \sim a_2 \iff a_1 \in [a_2] \iff a_2 \in [a_1]   \]
  %
  Somit sind beide Inklusionen gezeigt.
    %
  \item Seien $\sim$ und $C$ wie in Aufgabenteil \textbf{b)}. Ist $f : A \to B$ eine Abbildung und gilt $\forall a_1,a_2 \in A: (a_1 \sim a_2 \implies f(a_1)=f(a_2))$
so wird durch $g([a]_{\sim}) = f(a)$ für alle $a \in A$ eine Abbildung $g : C \to B$ definiert.
  %
  %
  Wir suchen also eine Abbildung $g: \{ [a]_{\sim} \given{a \in A} \} \to f(a) \forall A $.

  Aus \url{http://www.roeglin.org/teaching/WS2012/LuDS/LuDS.pdf}{ \textbf{2.12.}}

  \begin{definition}
  Eine Relation $f \subseteq A \times B $ heißt Abbildung oder Funktion, wenn
jedes $a \in A$ zu genau einem Element $b \in B$ in Relation steht. Um anzudeuten, dass $f$ eine Abbildung ist, schreiben wir $f: A \to B$ [...]
  \end{definition}

  Es muss also gezeigt werden, dass jedes $c \in C$ mit einem $b \in B$ in Relation steht.
  Wir betrachten $ g: [a] \to f(a) $ mit $f(a)$.  $f$ ist funktional fast äquivalent zu $p$, nur nicht bidirektional.
  Da wir uns aber nur für $g: C \to B$ interessieren ist dies keine Einschränkung.
  Damit gilt auch $C \subseteq B$.
  Weil $f$ ein eindirektionales $p$ ist gilt auch,
  dass $g([a]_{\sim}) = f(a)$
  für alle $a\in A$ mit $g:C\to B.$
  \end{enumerate}

  \subsection*{Aufgabe 2}

  \begin{enumerate}[a)]
    \item Beschreiben Sie für die Äquivalenzrelationen aus Aufgabe 2.a) und 2.c) vom Übungsblatt 3 die Äquivalenzklassen.
      \begin{enumerate}[i)]
        \item 2.a) $\abs{a} = \abs{b}$. Daher beispielsweise $[1] = \{-1, 1\}$ und $[2] = [-2, 2]$ Es gilt allgemein $[a] = \{ -a, a \} \; \forall a$.

        \item 2.c) $\{  (a,b) \in \Z \times \Z \given {\exists z \in \Z: \; a-b = z \cdot p}  \}$, für ein $p \in \N$.

        Für $z = 1, \; p=1$ findet man $[2] = \{1\}$ sowie $[3] = \{2\}$ usw. Allgemein gilt für hiermit $[n] = \{n-1\}$ für $n \in \Z$. Ebenso $[n] = n-p$ für $z$ fix und analog für $p$.


      \end{enumerate}


    \item Bestimmen Sie folgende Äquivalenzklassen:

    \begin{enumerate}[i)]
      \item $[42] \oplus_{47} [276]$\\
      Aus \url{http://www.roeglin.org/teaching/WS2012/LuDS/LuDS.pdf}{ \textbf{2.17.}}:

      \begin{definition}$[a] \oplus_{n} [b] = [a + b]_n$
      \end{definition}
      Daher suchen wir $[42 + 276]_47 = [318]_47$ Rest ist 36, daher gilt $[318]_47 = \{ x \, \text{mod} \, 47 = 36 \iff x = 47n + 36, \; n \in \Z  \} $

      \item $ [7] \odot_11 [19] $. Es folgt wieder $ [7 \cdot 19]_11 = [133]_11 $, Rest ist 1, daher gilt

      $ [133]_11 = \{\ x = 11n + 1, \; n \in \Z \iff [1] \; \text{mit} \; \equiv_{3} \} $.



    \end{enumerate}
  \end{enumerate}



