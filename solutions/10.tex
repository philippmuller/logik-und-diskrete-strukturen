\section*{Blatt 10}
%

\subsection*{Aufgabe 1}

Betrachte $Q = \{a + b \cdot \sqrt{2} \given{ a, b \in \Q}\} \subseteq \R.$

Zeigen Sie, dass $Q$ zusammen mit der Addition und Multiplikation aus $\R$ einen Körper bildet.

\begin{itemize}
  \item \textbf{Assoziativität}:

  \begin{align*}
    &((a_1 + b_1 \cdot \sqrt{2}) + (a_2 + b_2 \cdot \sqrt{2})) + (a_3 + b_3 \cdot \sqrt{2}) \\
    &\equiv (a_1 + b_1 \cdot \sqrt{2}) + (a_2 + b_2 \cdot \sqrt{2}) + (a_3 + b_3 \cdot \sqrt{2}) \\
    &\equiv a_1 + a_2 + a_3 + b_1 \cdot \sqrt{2} + b_2 \cdot \sqrt{2} + b_3 \cdot \sqrt{2} \\
    &\equiv (a_1 + b_1 \cdot \sqrt{2}) + ((a_2 + b_2 \cdot \sqrt{2}) + (a_3 + b_3 \cdot \sqrt{2}))
  \end{align*}

  \item \textbf{Neutrales Element}:

  Wir suchen $a + e = e + a = a$ für $a, e \in Q$.

  \[  a_1 + b_1 \sqrt{2} + (a_e + b_e \sqrt{2}) = a_1 + b_1 \sqrt{2}  \]

  Da 0 das neutrale Element der Addition ist können wir $a_e = b_e = 0$ wählen.

  \item \textbf{Inverses Element}:

  Es soll gelten, dass

  \[  x_1 + x_1^{-1} = e = 0, \; x_1, x_1^{-1} \in \Q  \]
 Daher

 \begin{align*}
   &\equiv (a_1 + b_2 \sqrt{2}) + a_1^{\star} + b_1^{\star}\sqrt{2} = 0 \\
   &\iff (a_1 + b_1 \sqrt{2}) - (a_1 + b_1 \sqrt{2}) = 0 \\
   &\implies (a_1, b_1)^{-1} = -(a_1, b_1) = (-a_1, -b_1) = x_1^{-1}
 \end{align*}

 \item \textbf{Kommutativität}

  \begin{align*}
    &x_1 + x_2 = x_2 + x_1, \; x_1, x_2 \in Q \\
    &\equiv (a_1 + b_1 \sqrt{2}) + (a_2 + b_2 \sqrt{2})\\
    & \text{Mit den Rechenregeln der Addition über $\Q$ haben wir} \\
    & (a_1 + b_1 \sqrt{2}) + (a_2 + b_2 \sqrt{2}) = (a_2 + b_2 \sqrt{2}) + (a_1 + b_1 \sqrt{2}) \\
  \end{align*}
  Damit haben wir eine abelsche Gruppe.

  \item \textbf{Assoziativität} der multiplikativen Verknüpfung $(Q, \cdot)$. Zu zeigen: $(x_1 \cdot x_2) \cdot x_3 = x_1 \cdot (x_2 \cdot x_3)$.

  \begin{align*}
  &((a_1 + b_1 \cdot \sqrt{2}) (a_2 + b_2 \cdot \sqrt{2})) \cdot (a_3  + b_3 \cdot \sqrt{2}) \\
    &\text{Mit Rechenregeln aus $\Q$ folgt}\\
    &\equiv (a_1 + b_1 \cdot \sqrt{2}) ((a_2 + b_2 \cdot \sqrt{2}) \cdot (a_3  + b_3 \cdot \sqrt{2}))
  \end{align*}
  Damit haben wir $(x_1 \cdot x_2) \cdot x_3 = x_1 \cdot (x_2 \cdot x_3)$.
  \item \textbf{Distributivität}

  Zu zeigen: $  x_1 \cdot (x_2 + x_3) = x_1 \cdot x_2 + x_1 \cdot x_3 , \; x_1, x_2, x_3 \in Q $

  \begin{align*}
    &(a_1 + b_1 \cdot \sqrt{2}) \cdot ((a_2 + b_2 \cdot \sqrt{2})+(a_3 + b_3 \cdot \sqrt{2} )) \\
    &\text{Mit den normalen Rechenregeln haben wir wieder} \\
    &\equiv (a_1 + b_1\cdot\sqrt{2})\cdot(a_2 + b_2 \cdot \sqrt{2}) +
      (a_1 + b_1 \cdot \sqrt{2}) \cdot (a_3 + b_3 \cdot \sqrt{2})\\
  \end{align*}


\end{itemize}

\subsection*{Aufgabe 2}

\begin{tabular}{c|cccccc}
    $\odot_7$
      & \dbr{1} & \dbr{2} & \dbr{3} & \dbr{4} & \dbr{5} & \dbr{6} \\
    \hline
    \dbr{1} & 1 & 2 & 3 & 4 & 5 & 6 \\
    \dbr{2} & 2 & 4 & 6 & 1 & 3 & 5 \\
    \dbr{3} & 3 & 6 & 2 & 5 & 1 & 4 \\
    \dbr{4} & 4 & 1 & 5 & 2 & 6 & 3 \\
    \dbr{5} & 5 & 3 & 1 & 6 & 4 & 2 \\
    \dbr{6} & 6 & 5 & 4 & 3 & 2 & 1 \\
    \end{tabular}\\[1em]


    Wie oben ersichtlich haben wie als erzeugendes Element \dbr{3} und \dbr{5}. Die Gruppe $(\Z / 7\Z, \odot_7)$ ist zyklisch, weil eben solche Erzeuger existieren. Wir haben nämlich
    \begin{align*}
      3^1 \mod 7 &= 3 \\
      3^2 \mod 7 &= 2 \\
      3^3 \mod 7 &= 6 \\
      3^4 \mod 7 &= 4 \\
      3^5 \mod 7 &= 5 \\
      3^6 \mod 7 &= 1 \\
    \end{align*}
    Damit haben wir ein $a \in G$ gefunden, so dass für alle $g \in G$ ein $j \in \Z$ existiert mit $g = a^j$.

\subsection*{Aufgabe 3}

  Seien $(R_1, +_1, \cdot_1)$ und $(R_2, +_2, \cdot_2)$ Ringe.

  \begin{enumerate}[a)]
    \item Zeigen Sie, dass $(R_1 \times R_2, +, \cdot)$ ein Ring ist

    \begin{itemize}
      \item \textbf{Abelsche Gruppe}, betrachte $(R_1 \times R_2, +), \; a_1, b_1 \in R_1, \; a_2, b_2 \in R_2$
      \begin{align*}
        (a_1, a_2) + (b_1, b_2) &= (a_1 +_1 b_1, a_2 +_2 b_2) \\
        (b_1, b_2) + (a_1, a_2) &= (b_1 +_1 a_1, b_2 +_2 a_2) \\
        &= (a_1 +_1 b_1, a_2 +_2 b_2) \\
      \end{align*}
      Die letzte Umformung gilt, da $(R_1, +_1)$ und $(R_2, +_2)$ und damit auch $(R_1, +_1) \land (R_2, +_2)$ abelsch sind.

      \item \textbf{Assoziativität}
      Seien $a_1, b_1, c_1 \in R_1$ und $a_2, b_2, c_2 \in R_2$.
      \begin{align*}
        ((a_1, a_2) \cdot (b_1, b_2)) \cdot (c_1, c_2) &= (a_1 \cdot_1 b_1, a_2 \cdot_2 b_2)\cdot (c_1, c_2) \\
        &= (a_1 \cdot_1 b_1 \cdot_1 c_1, a_2 \cdot_2 b_2 \cdot_2 c_2) \\
      \end{align*}
      Ebenso andersrum
      \begin{align*}
        (a_1, a_2) \cdot ((b_1, b_2) \cdot (c_1, c_2)) &= (a_1, a_2) \cdot (b_1 \cdot_1 c_1, b_2 \cdot_2 c_2) \\
        &= (a_1 \cdot_1 b_1 \cdot_1 c_1, a_2 \cdot_2 b_2 \cdot_2 c_2) \\
      \end{align*}

      \item \textbf{Distributivität}
        $a_1, b_1, c_1 \in R_1 \land a_2, b_2, c_2 \in R_2$.
        \begin{align*}
          (a_1, a_2) \cdot ((b_1, b_2)+(c_1, c_2)) &= (a_1, a_2)\cdot (b_1 +_1 c_1, b_2 +_2 c_2) \\
          &=(a_1 \cdot_1 (b_1 +_1 c_1 ), a_2 \cdot_2 (b_2 +_2 c_2))\\
          &=((a_1 \cdot_1 b_1) +_1 (a_1 \cdot_1 c_1), (a_2 \cdot_2 b_2) +_2 (a_2 \cdot_2 c_2)
        \end{align*}
        Somit ist $R_1 \times R_2, + , \times$ ein Ring.



    \end{itemize}
  \end{enumerate}

  \subsection*{Aufgabe 4}
  \begin{align*}
    x_1 = 6 \; \text{und} \; 17 \\
    x_2 = 4 \; \text{und} \; 13 \\
  \end{align*}

  Euklid:

  \begin{align*}
    17 &= 1 \cdot 13 + 4 \\
    13 &= 4 \cdot 3 + 1 \\
    3 &= 3 \cdot 1 + 0 \\
  \end{align*}

  ggT:

  \begin{align*}
    1 &= 13 - (4 \cdot 3) \\
      &= 13 - (17 - 13) \cdot 3 \\
      &= 13 - (3 \cdot 17 - 3 \cdot 13) \\
      &= 4 \cdot 13 - 3 \cdot 17 \\
      &\implies 4 \cdot 6\cdot 13 - 3 \cdot 4 \cdot 17 = 108\\
  \end{align*}


