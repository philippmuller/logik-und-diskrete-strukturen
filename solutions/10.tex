\section*{Blatt 10}
%

\subsection*{Aufgabe 1}

Betrachte $Q = \{a + b \cdot \sqrt{2} \given{ a, b \in \Q}\} \subseteq \R.$

Zeigen Sie, dass $Q$ zusammen mit der Addition und Multiplikation aus $\R$ einen Körper bildet.

\begin{itemize}
  \item Assoziativität:

  \begin{align*}
    &((a_1 + b_1 \cdot \sqrt{2}) + (a_2 + b_2 \cdot \sqrt{2})) + (a_3 + b_3 \cdot \sqrt{2}) \\
    &\equiv (a_1 + b_1 \cdot \sqrt{2}) + (a_2 + b_2 \cdot \sqrt{2}) + (a_3 + b_3 \cdot \sqrt{2}) \\
    &\equiv a_1 + a_2 + a_3 + b_1 \cdot \sqrt{2} + b_2 \cdot \sqrt{2} + b_3 \cdot \sqrt{2} \\
    &\equiv (a_1 + b_1 \cdot \sqrt{2}) + ((a_2 + b_2 \cdot \sqrt{2}) + (a_3 + b_3 \cdot \sqrt{2}))
  \end{align*}

  \item Neutrales Element:

  Wir suchen $a + e = e + a = a$ für $a, e \in Q$.

  \[  a_1 + b_1 \sqrt{2} + (a_e + b_e \sqrt{2}) = a_1 + b_1 \sqrt{2}  \]

  Da 0 das neutrale Element der Addition ist können wir $a_e = b_e = 0$ wählen.

  \item Inverses Element:

  Es soll gelten, dass

  \[  x_1 + x_1^{-1} = e = 0, \; x_1, x_1^{-1} \in \Q  \]
 Daher

 \begin{align*}
   &\equiv (a_1 + b_2 \sqrt{2}) + a_1^{\star} + b_1^{\star}\sqrt{2} = 0 \\
   &\iff (a_1 + b_1 \sqrt{2}) - (a_1 + b_1 \sqrt{2}) = 0 \\
   &\implies (a_1, b_1)^{-1} = -(a_1, b_1) = (-a_1, -b_1) = x_1^{-1}
 \end{align*}

 \item Kommutativität

  \begin{align*}
    &x_1 + x_2 = x_2 + x_1, \; x_1, x_2 \in Q \\
    &\equiv (a_1 + b_1 \sqrt{2}) + (a_2 + b_2 \sqrt{2})\\
    & \text{Mit den Rechenregeln der Addition über $\Q$ haben wir} \\
    & (a_1 + b_1 \sqrt{2}) + (a_2 + b_2 \sqrt{2}) = (a_2 + b_2 \sqrt{2}) + (a_1 + b_1 \sqrt{2}) \\
  \end{align*}
  Damit haben wir eine abelsche Gruppe.

  \item Assoziativität der multiplikativen Verknüpfung $(Q, \cdot)$. Zu zeigen: $(x_1 \cdot x_2) \cdot x_3 = x_1 \cdot (x_2 \cdot x_3)$.

  \begin{align*}
  &((a_1 + b_1 \cdot \sqrt{2}) (a_2 + b_2 \cdot \sqrt{2})) \cdot (a_3  + b_3 \cdot \sqrt{2}) \\
    &\text{Mit Rechenregeln aus $\Q$ folgt}\\
    &\equiv (a_1 + b_1 \cdot \sqrt{2}) ((a_2 + b_2 \cdot \sqrt{2}) \cdot (a_3  + b_3 \cdot \sqrt{2}))
  \end{align*}
  Damit haben wir $(x_1 \cdot x_2) \cdot x_3 = x_1 \cdot (x_2 \cdot x_3)$.
  \item Distributivität

  Zu zeigen: $  x_1 \cdot (x_2 + x_3) = x_1 \cdot x_2 + x_1 \cdot x_3 , \; x_1, x_2, x_3 \in Q $

  \begin{align*}
    &(a_1 + b_1 \cdot \sqrt{2}) \cdot ((a_2 + b_2 \cdot \sqrt{2})+(a_3 + b_3 \cdot \sqrt{2} )) \\
    &\text{Mit den normalen Rechenregeln haben wir wieder} \\
    &\equiv (a_1 + b_1\cdot\sqrt{2})\cdot(a_2 + b_2 \cdot \sqrt{2}) +
      (a_1 + b_1 \cdot \sqrt{2}) \cdot (a_3 + b_3 \cdot \sqrt{2})\\
  \end{align*}


\end{itemize}

\subsection*{Aufgabe 2}

\begin{tabular}{c|cccccc}
    $\odot_7$
      & \dbr{1} & \dbr{2} & \dbr{3} & \dbr{4} & \dbr{5} & \dbr{6} \\
    \hline
    \dbr{1} & 1 & 2 & 3 & 4 & 5 & 6 \\
    \dbr{2} & 2 & 4 & 6 & 1 & 3 & 5 \\
    \dbr{3} & 3 & 6 & 2 & 5 & 1 & 4 \\
    \dbr{4} & 4 & 1 & 5 & 2 & 6 & 3 \\
    \dbr{5} & 5 & 3 & 1 & 6 & 4 & 2 \\
    \dbr{6} & 6 & 5 & 4 & 3 & 2 & 1 \\

    \end{tabular}

    Wie oben ersichtlich haben wie als erzeugendes Element \dbr{3} und \dbr{5}. Die Gruppe $(\Z / 7\Z, \odot_7)$ ist zyklisch, weil eben solche Erzeuger existieren. Wir haben nämlich
    \begin{align*}
      3^1 \mod 7 &= 3 \\
      3^2 \mod 7 &= 2 \\
      3^3 \mod 7 &= 6 \\
      3^4 \mod 7 &= 4 \\
      3^5 \mod 7 &= 5 \\
      3^6 \mod 7 &= 1 \\
    \end{align*}
    Damit haben wir ein $a \in G$ gefunden, so dass für alle $g \in G$ ein $j \in \Z$ existiert mit $g = a^j$.