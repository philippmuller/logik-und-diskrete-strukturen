\section*{Blatt 10}
%

\subsection*{Aufgabe 1}

Betrachte $Q = \{a + b \cdot \sqrt{2} \given{ a, b \in \Q}\} \subseteq \R.$

Zeigen Sie, dass $\Q$ zusammen mit der Addition und Multiplikation aus $\R$ einen Körper bildet.

\begin{itemize}
  \item Assoziativität:

  \begin{align*}
    &((a_1 + b_1 \cdot \sqrt{2}) + (a_2 + b_2 \cdot \sqrt{2})) + (a_3 + b_3 \cdot \sqrt{2}) \\
    &\equiv (a_1 + b_1 \cdot \sqrt{2}) + (a_2 + b_2 \cdot \sqrt{2}) + (a_3 + b_3 \cdot \sqrt{2}) \\
    &\equiv a_1 + a_2 + a_3 + b_1 \cdot \sqrt{2} + b_2 \cdot \sqrt{2} + b_3 \cdot \sqrt{2} \\
    &\equiv (a_1 + b_1 \cdot \sqrt{2}) + ((a_2 + b_2 \cdot \sqrt{2}) + (a_3 + b_3 \cdot \sqrt{2}))
  \end{align*}

  \item Neutrales Element:

  Wir suchen $a + e = e + a = a$ für $a, e \in \Q$.

  \[  a_1 + b_1 \sqrt{2} + (a_e + b_e \sqrt{2}) = a_1 + b_1 \sqrt{2}  \]

  Da 0 das neutrale Element der Addition ist können wir $a_e = b_e = 0$ wählen.

  \item Inverses Element:

  Es soll gelten, dass

  \[  x_1 + x_1^{-1} = e = 0, \; x_1, x_1^{-1} \in \Q  \]
 Daher

 \begin{align*}
   &\equiv (a_1 + b_2 \sqrt{2}) + a_1^{\star} + b_1^{\star}\sqrt{2} = 0 \\
   &\iff (a_1 + b_1 \sqrt{2}) - (a_1 + b_1 \sqrt{2}) = 0\\
   &\implies (a_1, b_1)^{-1} = -(a_1, b_1) = (-a_1, -b_1) = x_1^{-1}
 \end{align*}

 \item Kommutativität

  \begin{align*}
    &x_1 + x_2 = x_2 + x_1, \; x_1, x_2 \in \Q \\
    &\equiv (a_1 + b_1 \sqrt{2}) + (a_2 + b_2 \sqrt{2})\\
    & \text{Mit den Rechenregeln der Addition über $\Q$ haben wir} \\
    & (a_1 + b_1 \sqrt{2}) + (a_2 + b_2 \sqrt{2}) = (a_2 + b_2 \sqrt{2}) + (a_1 + b_1 \sqrt{2}) \\
  \end{align*}
  Damit haben wir eine abelsche Gruppe.

  \item Assoziativität der multiplikativen Verknüpfung $(a, \cdot)$

  \item Distributivität

\end{itemize}